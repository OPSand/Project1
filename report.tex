%% LyX 2.0.6 created this file.  For more info, see http://www.lyx.org/.
%% Do not edit unless you really know what you are doing.
\documentclass[english]{article}
\usepackage[T1]{fontenc}
\usepackage{listings}
\setlength{\parskip}{\smallskipamount}
\setlength{\parindent}{0pt}
\usepackage{amssymb}
\usepackage{setspace}
\doublespacing
\usepackage{babel}
\usepackage{xunicode}
\begin{document}

\title{Project 1, FYS 3150 / 4150, fall 2013}


\author{Nathalie Bonatout and Odd Petter Sand}

\maketitle

\section{Introduction}

In this project we will solve the one-dimensional Poissson equation
with Dirichlet boundary conditions by rewriting it as a set of linear
equations.

To be more explicit we will solve the equation 
\[
-u''(x)=f(x),\hspace{0.5cm}x\in(0,1),\hspace{0.5cm}u(0)=u(1)=0.
\]
and we define the discretized approximation to $u$ as $v_{i}$ with
grid points $x_{i}=ih$ in the interval from $x_{0}=0$ to $x_{n+1}=1$.
The step length or spacing is defined as $h=1/(n+1)$. We have then
the boundary conditions $v_{0}=v_{n+1}=0$. We approximate the second
derivative of $u$ with 
\[
-\frac{v_{i+1}+v_{i-1}-2v_{i}}{h^{2}}=f_{i}\hspace{0.5cm}\mathrm{for}\hspace{0.1cm}i=1,\dots,n,
\]
where $f_{i}=f(x_{i})$.

(Author's note: This text, and the text introducing the various exercises,
is taken from the project description provided at the course website.)


\section{Exercises}


\subsection{Exercise a)}

We start with the given equation

$-\frac{v_{i-1}-2v_{i}+v_{i+1}}{h^{2}}=f_{i}$

where $i\in[1,n]\cap\mathbb{N}$ and $n\in\mathbb{N}$. We will assume
that $n\geqslant3$ and that $v_{0}=v_{n+1}=0$.

$-v_{i-1}+2v_{i}-v_{i+1}=h^{2}f_{i}\equiv\widetilde{b}_{i}$

$\left[\begin{array}{ccc}
-1 & 2 & -1\end{array}\right]\left[\begin{array}{c}
v_{i-1}\\
v_{i}\\
v_{i+1}
\end{array}\right]=\widetilde{b}_{i}$

Now we expand these vectors from 3 elements to $n$ elements. Note
that the $i$th element of the row vector should be 2 and the $i$th
element of the column vector should be $v_{i}$ after the expansion:

$\left[\begin{array}{ccccccc}
0 & \cdots & -1 & 2 & -1 & \cdots & 0\end{array}\right]\left[\begin{array}{c}
v_{1}\\
\vdots\\
v_{i-1}\\
v_{i}\\
v_{i+1}\\
\vdots\\
v_{n}
\end{array}\right]=\widetilde{b}_{i}$

Note that in the row vector 2 can very well be the first or last element,
in which case the preceding presentation can be a little misleading.
We further recognize the row vector as the $i$th row of the $n\times n$
matrix $\mathbf{A}$ (defined such that $A_{ij}=2\delta_{ij}-\delta_{i(j-1)}-\delta_{i(j+1)}$
) and get the inner product

$\mathbf{A}_{i}\cdot\mathbf{v}=\widetilde{b}_{i}$

and remembering that $i\in[1,n]$, by the definition of matrix multiplication

$\mathbf{A}\mathbf{v}=\mathbf{\widetilde{b}}$

which is what we wanted to show.


\subsection{Exercise b)}

The algorithm we will use is as follows: First we do forward subsitution
by subtracting from the current row a multiple of the row before it. 

$A_{i}=A_{i}-x_{i}A_{i-1}$

Here, $x_{i}$ is the factor that cause the term $a_{i}$ to cancel
out. Before the first step, the rows $i-1$ and $i$ look like this
(the {*} indicates a value that has been changed by the algoritm)

$\begin{array}{ccccccc}
\cdots & 0 & b_{i-1}^{*} & c_{i-1} & 0 & 0 & \cdots\\
\cdots & 0 & a_{i} & b_{i} & c_{i} & 0 & \cdots
\end{array}$

and our goal is that it look like this after the forward substitution

$\begin{array}{ccccccc}
\cdots & 0 & b_{i-1}^{*} & c_{i-1} & 0 & 0 & \cdots\\
\cdots & 0 & 0 & b_{i}^{*} & c_{i} & 0 & \cdots
\end{array}$

(note that $c_{i}$ is unchanged). When we reach the bottom, we do
a backward subsitution by adding to the current row a multiple of
the row below it and then dividing the last element by itself to make
the last element equal to 1:

$A_{i}=A_{i}-c_{i}A_{i+1}$

$A_{i}=\frac{A_{i}}{b_{i}^{*}}$

That is to say, we go from

$\begin{array}{cccccc}
\cdots & 0 & b_{i}^{*} & c_{i} & 0 & \cdots\\
\cdots & 0 & 0 & 1 & 0 & \cdots
\end{array}$

to

$\begin{array}{cccccc}
\cdots & 0 & 1 & 0 & 0 & \cdots\\
\cdots & 0 & 0 & 1 & 0 & \cdots
\end{array}$

Naturally, we also have to do equivalent operations on the vector
\textbf{$\mathbf{\widetilde{b}}$} on the other side of the equation.
We will not calculate any unnecessary values, i.e. values that will
not be used by the program later on. Hence the algoritm looks like
this (number of flops in parantheses):

For $i$ : $2\longrightarrow n$
\begin{enumerate}
\item $x_{i}=\frac{a_{i}}{b_{i-1}^{*}}\;$ ($n$)
\item $b_{i}=b_{i}-x_{i}c_{i-1}\;$($2n$)
\item $\widetilde{b}_{i}=\widetilde{b}_{i}-x_{i}\widetilde{b}_{i-1}\;$($2n$)
\end{enumerate}
$\widetilde{b}_{n}=\frac{\widetilde{b}_{n}}{b_{n}}\;(1)$

For $i$ : $n-1\longrightarrow1$
\begin{enumerate}
\item $\widetilde{b}_{i}=\widetilde{b}_{i}-c_{i}\widetilde{b}_{i+1}\;$($2n$)
\item $\widetilde{b}_{i}=\frac{\widetilde{b}_{i}}{b_{i}}$$\;$ ($n$)
\end{enumerate}
This makes the total running time of the algorihm $8n$ (actually,
it is $8(n-1)$, but we are mostly concerned with the performance
for large $n$, where $n\approx n-1$).

If the matrix is symmetric, so that $a_{i}=c_{i-1},$ then $x_{i}c_{i-1}=\frac{a_{i}^{2}}{b_{i-1}^{*}}$.

If $a_{i}=c_{i-1}=k$ for all $i$, we have a special case where $k=\pm1$.
Then $x_{i}=\pm\frac{1}{b_{i-1}^{*}}$, $k^{2}=1$ and $x_{i}c_{i-1}=\frac{1}{b_{i-1}^{*}}=-x_{i}\equiv y_{i}$.

Furthermore, $c_{i}\widetilde{b}_{i+1}=k\widetilde{b}_{i+1}=\mp\widetilde{b}_{i+1}$
(note: opposite of the sign of $k$). The algorithm will then look
like this for $k=-1$:

For $i$ : $2\longrightarrow n$
\begin{enumerate}
\item $y_{i}=\frac{1}{b_{i-1}^{*}}\;$ ($n$)
\item $b_{i}=b_{i}-y_{i}\;$($n$)
\item $\widetilde{b}_{i}=\widetilde{b}_{i}+y_{i}\widetilde{b}_{i-1}\;$($2n$)
\end{enumerate}
$\widetilde{b}_{n}=\frac{\widetilde{b}_{n}}{b_{n}}\;(1)$

For $i$ : $n-1\longrightarrow1$
\begin{enumerate}
\item $\widetilde{b}_{i}=\widetilde{b}_{i}+\widetilde{b}_{i+1}\;$($n$)
\item $\widetilde{b}_{i}=\frac{\widetilde{b}_{i}}{b_{i}}$$\;$ ($n$)
\end{enumerate}
This gives a total running time of $6n$ for this implementation of
the algorithm, which we ended up using for this project.

<INSERT RESULTS + PLOTS HERE>


\subsection{Exercise c)}

In thie exercise we will compute the relative error in the data set
$i=1,\dots,n$,by setting up 
\[
\epsilon_{i}=log_{10}\left(\left|\frac{v_{i}-u_{i}}{u_{i}}\right|\right),
\]
as function of $log_{10}(h)$ for the function values $u_{i}$ and
$v_{i}$. For each step length we will extract the max value of the
relative error.

...


\subsection{Exercise d)}

(TODO: Change the text below to sound less like an exercise and more
like a description.)

Compare your results with those from the LU decomposition codes for
the matrix of sizes $10\times10$, $100\times100$ and $1000\times1000$.
Here you should use the library functions provided on the webpage
of the course. Use for example the unix function {\em time} when
you run your codes and compare the time usage between LU decomposition
and your tridiagonal solver. Alternatively, you can use the functions
in C++, Fortran or Python that measure the time used.

Make a table of the results and comment the differences in execution
time How many floating point operations does the LU decomposition
use to solve the set of linear equations? Can you run the standard
LU decomposition for a matrix of the size $10^{5}\times10^{5}$? Comment
your results.

To compute the elapsed time in c++ you can use the following statements
\begin{lstlisting}[title={Time in C++}]
using namespace std;
...
#include "time.h"   //  you have to include the time.h header
int main()
{
    // declarations of variables 
    ...
    clock_t start, finish;  //  declare start and final time
    start = clock();
    // your code is here, do something and then get final time
    finish = clock();
    ( (finish - start)/CLOCKS_PER_SEC );
...
\end{lstlisting}


...


\subsection{Exercise e)}

In this exercise we are investigating matrix multiplication in row
major order versus column major order with regards to running time.

The task here is to write a small program which sets up two random
(use the ran0 function in the library lib.cpp to initialize the matrix)
double precision valued matrices of dimension $5000\times5000$. (NOTE:
The original value of $10^{4}\times10^{4}$ proved too memory intensive
for our poor laptops.)

The multiplication of two matrices ${\bf A}={\bf BC}$ could then
take the following form in standard row-major order \lstset{language=c++}
\begin{lstlisting}
      for(j=0 ; j < n ; j++) {
         for(k=0 ; k < n ; k++) {
            a[i][j]+=b[i][k]*c[k][j]
         }
      }
   }  
\end{lstlisting}
 and in a column-major order as \lstset{language=c++} 
\begin{lstlisting}
      for(i=0 ; i < n ; i++) {
         for(k=0 ; k < n ; k++) {
            a[i][j]+=b[i][k]*c[k][j]
         }
      }
   }  
\end{lstlisting}


(NOTE: We implemented this using dynamic memory allocation, as lib.cpp
proved difficult to add to the project in Visual C++.)


\section{Conclusion}

In this project we learned that...


\subsection{Critique}

We would like to provide the following items of feedback for future
versions of this project:
\begin{itemize}
\item lib.cpp sucks!
\item we need more RAM. you made our laptops feel sad.
\item ???\end{itemize}

\end{document}
